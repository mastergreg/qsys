\documentclass[a4paper,11pt]{article} \usepackage{anysize}
\marginsize{2cm}{2cm}{1cm}{1cm}
%\textwidth 6.0in \textheight = 664pt
\usepackage{xltxtra}
\usepackage{xunicode}
\usepackage{graphicx}
\usepackage{color}
\usepackage[table]{xcolor}
%\usepackage[usenames,dvipsnames]{xcolor}
\usepackage{xgreek}
\usepackage{fancyvrb}
\usepackage{minted}
\usepackage{listings}
\usepackage{enumitem} \usepackage{framed} \usepackage{relsize}
\usepackage{float} 

\definecolor{bootred}{RGB}{248,148,6}
\definecolor{bootgreen}{RGB}{81,163,81}
\definecolor{bootblue}{RGB}{73, 178, 205}


\renewcommand{\theenumi}{\roman{enumi}}

\def\thesection {Ζητούμενο \arabic{section}}

\setmainfont[Mapping=TeX-text]{CMU Concrete}


\begin{document}

\begin{titlepage}
    \begin{center}
        \begin{figure}[t] 
            \includegraphics[scale=0.7]{title/ntua_logo}
        \end{figure}
        \begin{LARGE}\textbf{ΕΘΝΙΚΟ ΜΕΤΣΟΒΙΟ ΠΟΛΥΤΕΧΝΕΙΟ\\}\end{LARGE}
        \vspace{2cm}
        \begin{Large}
            ΣΧΟΛΗ ΗΜ\&ΜΥ\\
            Συστήματα Αναμονής\\
            Άσκηση Προσομοίωσης\\
            Ακ. έτος 2012-2013\\
        \end{Large}
        \vfill
        \begin{tabular}{l r}
            \Large{Λύρας Γρηγόρης}&
            \large{Α.Μ.: 03109687}\\
        \end{tabular}\\
        \vspace{5cm}

        \vfill
        \large\today\\
    \end{center}
\end{titlepage}







\pagebreak
\section*{Εισαγωγή}

Στην εργασία αυτή προσομοιώσαμε ένα σύστημα αναμονής M/M/1/12 με κατώφλι
buffer Κ. Οι αφίξεις ακολοθούν ρυθμό Poisson, οι εξυπηρετήσεις εκθετικο,
έχουμε έναν εξυπηρετητή και μέγιστο αριθμό πελατών 12. Το κατώφλι του
streaming buffer είναι $1 \leq K \leq 11 $

\section{Η πιθανότητα το σύστημα να βρίσκεται στην κατάσταση s, $P_s(K)$, για όλες τις
δυνατές καταστάσεις του συστήματος s και τις τιμές του κατωφλίου Κ.}

\begin{figure}[H]
    \centering
    \includegraphics[width=\textwidth]{files/3_a_propabilities.pdf}
    \caption{λ = 3, K=0,1,2,3,4,5}
\end{figure}

\begin{figure}[H]
    \centering
    \includegraphics[width=\textwidth]{files/3_b_propabilities.pdf}
    \caption{λ = 3, K=6,7,8,9,10,11}
\end{figure}

\begin{figure}[H]
    \centering
    \includegraphics[width=\textwidth]{files/5_a_propabilities.pdf}
    \caption{λ = 5, K=0,1,2,3,4,5}
\end{figure}

\begin{figure}[H]
    \centering
    \includegraphics[width=\textwidth]{files/5_b_propabilities.pdf}
    \caption{λ = 5, K=6,7,8,9,10,11}
\end{figure}

\begin{figure}[H]
    \centering
    \includegraphics[width=\textwidth]{files/12_a_propabilities.pdf}
    \caption{λ = 12, K=0,1,2,3,4,5}
\end{figure}

\begin{figure}[H]
    \centering
    \includegraphics[width=\textwidth]{files/12_b_propabilities.pdf}
    \caption{λ = 12, K=6,7,8,9,10,11}
\end{figure}

\section{Η πιθανότητα υπερχείλισης του συστήματος $P_N(K)$ σαν συνάρτηση του
    κατωφλίου Κ.}

\begin{figure}[H]
    \centering
    \includegraphics[width=\textwidth]{files/3_2_propabilities.pdf}
    \caption{λ = 3}
\end{figure}

\begin{figure}[H]
    \centering
    \includegraphics[width=\textwidth]{files/5_2_propabilities.pdf}
    \caption{λ = 5}
\end{figure}

\begin{figure}[H]
    \centering
    \includegraphics[width=\textwidth]{files/12_2_propabilities.pdf}
    \caption{λ = 12}
\end{figure}

\section{Ο μέσος αριθμός πακέτων στο σύστημα $E_N(K)$ σαν συνάρτηση του κατωφλίου Κ.}
\[
    E_N = \sum_{i=0}^{11}i*P_{i}
\]
\begin{figure}[H]
    \centering
    \includegraphics[width=\textwidth]{files/e_n.pdf}
    \caption{$E_N(K)$}
\end{figure}

\section{H μέση ρυθμαπόδοση του συστήματος γ(Κ) σαν συνάρτηση του κατωφλίου Κ.}
\[
    \gamma = \lambda \cdot (1-P_{bl})
\]

\begin{figure}[H]
    \centering
    \includegraphics[width=\textwidth]{files/g.pdf}
    \caption{γ(Κ)}
\end{figure}

\section{H μέση καθυστέρηση πακέτου στο σύστημα $T_d(K)$ σαν συνάρτηση του
    κατωφλίου Κ.}
\[
    E[T] = \frac{E[N]/\gamma}
\]
\begin{figure}[H]
    \centering
    \includegraphics[width=\textwidth]{files/t.pdf}
    \caption{T(Κ)}
\end{figure}

\section*{Παρατηρήσεις}
\begin{itemize}
        \item Για Κ = 1 το σύστημα είναι μια τυπική ουρά Μ/Μ/1/12. 
        \item Παρατηρήσαμε πως για μικρό λ η ουρά δε γεμίζει συνεπώς το
            μεγαλύτερο μέρος του χρόνου του συστήματος καταναλώνεται στο
            buffering. Αυτό φαίνεται και από το πολύ μικρό $P_{bl}$.
        \item Για λ = 12 έχουμε $\frac{\labmda}{\mhi} > 1$ συνεπώς εκεί
            εμφανίζεται μεγάλη πιθανότητα απόρριψης.
\end{itemize}

\pagebreak
\section*{Πηγαίος Κώδικας}

\inputminted[linenos,fontsize=\scriptsize,frame=leftline]{cpp}{files/main.cpp}

\end{document}
